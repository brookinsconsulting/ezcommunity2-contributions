%% LyX 1.1 created this file.  For more info, see http://www.lyx.org/.
%% Do not edit unless you really know what you are doing.
\documentclass[10pt,oneside,english,british]{book}
\usepackage{times}
\usepackage[T1]{fontenc}
\usepackage[latin1]{inputenc}
\usepackage{a4wide}
\usepackage{fancyhdr}
\pagestyle{fancy}
\usepackage{babel}
\setcounter{secnumdepth}{3}
\setcounter{tocdepth}{3}
\setlength\parskip{\smallskipamount}
\setlength\parindent{0pt}
\usepackage{graphics}
\IfFileExists{url.sty}{\usepackage{url}}
                      {\newcommand{\url}{\texttt}}

\makeatletter

%%%%%%%%%%%%%%%%%%%%%%%%%%%%%% LyX specific LaTeX commands.
\providecommand{\LyX}{L\kern-.1667em\lower.25em\hbox{Y}\kern-.125emX\@}

%%%%%%%%%%%%%%%%%%%%%%%%%%%%%% User specified LaTeX commands.
\renewcommand{\headrulewidth}{0.4pt}
\renewcommand{\footrulewidth}{0.4pt}
\lhead{\resizebox{1in}{!}{\includegraphics{screenshots/2.0/ezsystems.eps}}}
\rhead{}
\chead{}

\makeatother
\AtBeginDocument{
  \renewcommand{\labelitemiv}{}
}

\begin{document}

\title{eZ publish 2.2 Installation Guide}


\author{\resizebox*{0.75\columnwidth}{!}{\includegraphics{screenshots/2.0/ezpublish.eps}} }

\maketitle
\resizebox*{!}{0.2in}{\includegraphics{screenshots/2.0/ezsystems.eps}} The
double squares and eZ are trademarks belonging to eZ systems of Norway,
registration number NO 981 601 564 (http://www.brreg.no/oppslag/enhet/detalj.ssc?orgnr=981601564).

All images and text herein is Copyright 2001 eZ systems.

eZ publish is a software package released under the GPL lisence (http://www.gnu.org/copyleft/gpl.html),
its primary point of distribution and information is http://developer.ez.no/

\tableofcontents{}


\chapter{Introduction\label{chptr: introduction}}

\begin{quote}
\textbf{{}``He who asks is a fool for five minutes, but he who does
not ask remains a fool forever.''} - \-
\end{quote}
eZ publish is a content management system, among a lot of other things.
This installation manual will try to cover the job of installing eZ
publish on your server.

This manual is mainly intended for installation on a Red Hat Linux
system, but a lot of friendly people have contributed information
for installation on other operating systems, take a look at chapter
2 and learn which systems those are.

Most of what is described here regarding Red Hat installation can
also be applied to other installations, especially if your system
uses RPM for installation. For other systems you would need to do
a lot of compiling yourself to make this work, or apply the system's
own package manager.

Finding packages can be done dirctly from vendor sites, though you
might not be guaranteed that you'll find the package you need. In
such instances you need to download the source directly from the software
developer.

Different distribution sites for different Unix systems are:

\begin{itemize}
\item Debian \texttt{\footnotesize (http://www.debian.org/distrib/ftplist)}{\footnotesize \par}
\item Mandrake, see chapter \ref{chptr: mandrake}.
\item IRIX \texttt{\footnotesize (http://freeware.sgi.com/)}{\footnotesize \par}
\item Red Hat Linux \texttt{\footnotesize (http://www.redhat.com/apps/download})
\item SuSE Linux (\texttt{\footnotesize http://www.suse.com/us/support/download/index.html)}{\footnotesize \par}
\item Sun \texttt{\footnotesize (http://www.sunfreeware.com/)}{\footnotesize \par}
\end{itemize}
The addresses to the software developers will be given where apropriate
in the text.

You can also try \char`\"{}The Written Word\char`\"{} (\texttt{\footnotesize ftp://ftp.thewrittenword.com/packages/free/by-name/gcc-2.95.2/})
for binaries for Solaris 2.5.1, 2.6, 2.7/SPARC, 2.7/Intel, IRIX 6.2,
6.5, Digital UNIX 4.0D, HP-UX 10.20, and HP-UX.


\section{Pre-Configured Hosting}

It is possible to get pre-configured hosting services where you can
install and manage your eZ publish site with ease. Read more about
our hosting partners at eZ systems web site (\texttt{\footnotesize http://en.ez.no/article/articlestatic/73}).


\section{Pre-Configured Hardware}

It is possible to order pre-configured hardware from eZ systems. You
can order through or web shop (\texttt{\footnotesize http://sourceprovider.com/}).

A line starting with a hash-sign {}``\#'' are input from the user
to the shell.


\chapter{Pre-requisites\label{chptr: pre-requisites}}


\section{Needed Privileges}

For the standard installation (and for the moment the only method)
of eZ publish you will need to have the following privileges on your
system:

\begin{itemize}
\item Access to Apache's httpd.conf for creating two virtual hosts and for
enabling the rewrite engine and creating rewrite rules. This is absolutely
necessary for eZ Publish at the moment.
\item Access to compiler, only needed if you can't use any of the pre-compiled
packages available. (You will have to install the gcc compiler on
your system, see chapter \ref{chptr: introduction} for a list of
sites providing software for different Unixes.)
\item Access to a shell (You must run certain scripts during installation,
and sometimes for maintenance.)
\item Access to cron jobs (Only needed if you want to use the eZ news feed
module for regular updates of headlines imported from other sites.)
\item Access to Apache's modules
\item Access to a MySQL database
\item You might also need the privilege to add new libraries to your system.
\end{itemize}
You might also use other web servers than apache, but then you're
on your own since we haven't tested eZ publish on other configurations.
If you do try another web server, please keep a log of what you do
and submit it to us (pkej@ez.no) for inclusion in future versions
of this manual.


\section{Needed Software}

You also need to download and install the following packages, if they
aren't present on your system already:

\begin{itemize}
\item MySQL (http://www.mysql.com) version 3.23 or later. (eZ publish requires
MySQL for storage of its data.)
\item libXml2 (http://xmlsoft.org/\#Downloads) version 2.4.1 is recommended
but versions as old as 2.2.9 is known to work. (Needed by eZ article.
If you wish to use the default article renderer you need libXml2 installed.
You can create your own renderers if you don't want to use the default.)
\item libQdom (http://www.trolltech.com) is a part of QT, you need version
2.2.3 or later. (Needed by eZ news feed's parsers. If you wish to
include headlines from external sites (example developer.ez.no or
slashdot.org) then you need this installed. You can create your own
parsers if you don't want to use the default.)
\item ImageMagick (http://www.imagemagick.org/) newest version (Needed by
eZ article, eZ image catalogue, and all modules using images. You
need only the command line version.)
\item Apache (http://httpd.apache.org/) latest 1.3 release. (It is always
recommended to run the latest Apache release, though eZ publish shouldn't
be very picky with the Apache versions. We've used eZ publish with
Apache 1.3.13, some have reported that Apache 1.3.9 isn't useful.)
\item mod\_rewrite. This apache module is included in all recent versions
of RedHat Linux. If you use an other distro, you may need to recompile
apache with mod\_rewrite
\item Any and all modules you need for apache in addition to mod\_php. (http://modules.apache.org/)
\item PHP (http://www.php.net/) version 4.0.4pl1 or later. Version 4.0.6
is recommended. You need the source code version from this site, for
windows you can just download the binary. (eZ publish uses references
for objects and foreach loops. Only version 4.0.4pl1 and later supports
both of these features satisfactorily.)
\item eZ publish (http://developer.ez.no/) verision 2.0 or later stable
releases.
\end{itemize}
The libraries and php are packaged pre-compiled for Linux i386 on
http://developer.ez.no. The software is listed in the order of installation.

You should also find a list of RPMs at http://www.brandish.co.uk/phprpm


\section{Which Software is Already Installed?}


\subsection{Systems Using RPM}

RPM is a system for distributing pre-compiled software. The packages
also contain pre-configured settings and initialisation files, leaving
almost nothing to the user, except deciding what to install.

To check if a package is available on your system you can run the
following command (RPM based systems {}``rpm -qa | grep <name of
program/library>''. If you need to know where you can find the different
files from that package you can follow up on the previous command
with the following {}``rpm -ql <rpm name>''. RPM name is one of
the returned names from the previous command, example:\\


\texttt{\footnotesize ~~~~\# pkej@vogol:/etc/httpd > rpm -qa |
grep libxml}{\footnotesize \par}

\texttt{\footnotesize ~~~~libxml-1.8.7-80}{\footnotesize \par}

\texttt{\footnotesize ~~~~libxmld-1.8.7-80}{\footnotesize \par}

\texttt{\footnotesize ~~~~\# pkej@vogol:/etc/httpd > rpm -ql libxml-1.8.7-80}{\footnotesize \par}

\texttt{\footnotesize ~~~~/usr/bin/xml-config}{\footnotesize \par}

\texttt{\footnotesize ~~~~/usr/lib/libxml.so.1}{\footnotesize \par}

\texttt{\footnotesize ~~~~/usr/lib/libxml.so.1.8.7}{\footnotesize \par}

\texttt{\footnotesize ~~~~/usr/share/doc/packages/libxml}{\footnotesize \par}

\texttt{\footnotesize ~~~~/usr/share/doc/packages/libxml/AUTHORS}{\footnotesize \par}

\texttt{\footnotesize ~~~~/usr/share/doc/packages/libxml/COPYING}{\footnotesize \par}

\texttt{\footnotesize ~~~~/usr/share/doc/packages/libxml/COPYING.LIB}{\footnotesize \par}

\texttt{\footnotesize ~~~~/usr/share/doc/packages/libxml/NEWS}{\footnotesize \par}

\texttt{\footnotesize ~~~~/usr/share/doc/packages/libxml/README}{\footnotesize \par}

\texttt{\footnotesize ~~~~/usr/share/doc/packages/libxml/TODO}{\footnotesize \par}


\section{FreeBSD}

When installing and compiling PHP on a FreeBSD system you might encounter
an error when using --with-dom which says you have a conifgure error
on the lib. It turns out that the current port of libxml installs
itself as /usr/local/lib/libxml2.a|so and it goes unrecognised by
configure. You can easily get around this problem by linking the libs
to libxml.a|so.


\section{Mandrake}

First read chapter \ref{chptr: mandrake}, then continue reading the
manual from here.


\section{IRIX}

By accessing the software manager (you must be root) you can get a
list of installed software, scroll or search that list to find the
packages you're interested in. Double click on the tabs to the left
to get information about where specific files are installed.


\section{RAQ 3}

There is a separate chapter \ref{chptr: raq 3} in this manual describing
installation on a RAQ 3 server. It was kindly provided by Chris Mason, 


\section{Windows}

Windows installation is described in its own chapter \ref{chptr: windows}.


\section{Other Systems}

On other systems you should read the documentation for that system
to learn how to find out what software is already installed.

You could try to use the command {}``find'' to find the software.
It is used thus: {}``find . -name \textbackslash{}{*}<program name>\textbackslash{}{*}''
from the /usr/, /local/ , /lib/, /share/ directories. In extreme cases
you could try from the root of the system, but this will take a long
time and will also hog resources on your computer. Therefore we urge
you to learn how to use the proper installation features of your system
to find the software already installed.


\section{Installation of Required Software}

If you've found pre-compiled versions of all the software packaged
for use with an installation tool, you just have to install that software
using the tool. Instructioins for its usage is often found using the
command {}``man <installation tool name>'' or by reading your system's
documentation or the supplier's website.

If you've had to download source code you will find instructions on
how to compile and install the software you've downloaded at the software
developer's website. This requires a bit of knowledge and you should
only undertake this if you feel confident about the job.

This manual will only cover configuration of the software needed and
compilation of PHP to use the other software.


\section{Important Notice}

You should read all the README, INSTALL and similar files found with
the software packages you download. They often contain tips on how
to configure, compile and install the software on your system. It
will save you a lot of time and aggravation if you follow instructions
supplied with the software.

If problems arise during installation of the software, please turn
to the suppliers support forums, mailing list archives and FAQs, your
questions will often be answered there. If the supplier's forums doesn't
seem to help you, you should check the support forums at our site.

You should always do a search of the forums before posting any questions.


\chapter{Compile Configuration}


\section{PHP}

Important : YOU NEED TO RECOMPILE PHP. No known Linux distros does
yet have all the php features required by eZ~publish. This means
that you need to compile the php module from source.\\
You may find precompiled binaries for your system at the eZ publish web site,  \url{http://developer.ez.no}.
Take a look at the {}``Contributions'' section in the download area.


\subsection{Unpacking}

After you have downloaded PHP you need to unpack it somewhere where
you can compile and configure the software. To unpack run the command:\\


\texttt{\footnotesize ~~~~\# tar zxvf php-4.0.x.tar.gz}~\\
{\footnotesize \par}

Where the x is the version of php you've downloaded. Then you need
to move into the directory you extracted php into:\\


\texttt{\footnotesize ~~~~\# cd php-4.0.x}{\footnotesize \par}


\subsection{Configuration}

You'll need either an apache module or a command line (CGI) version
of PHP to use eZ publish on your website. We recommend you use PHP
as an apache module. You will also need the command line version if
you want to use the cron jobs for periodical updates of the eZ news
feed module.

Thus for our recommended installation of PHP you need both the command
line and module versions of PHP.


\subsubsection{Common}

Both the command line and apache module versions need to have the
following configurations added to the configuration tool:

\begin{description}
\item [--enable-trans-sid]This lets PHP use session id's which don't rely
on cookies. It does not disable normal cookie based sessions.\\
({\footnotesize http://www.php.net/manual/en/install.configure.php\#install.configure.enable-trans-sid})
\item [--with-mysql]This tells PHP that the mysql functionality should be
used.\\
({\footnotesize http://www.php.net/manual/en/install.configure.php\#install.configure.with-mysql})
\item [--disable-magic-quotes]This tells PHP to disable magic quotes by
default. you can also turn this feature on and off on a directory
by directory basis in either the {}``.htaccess'' files (if you use
them) or in the setup of the virtual server in {}``httpd.conf''. 
\item [IMORTANT]: From version 2.1 onwards magic quotes must be turned off
for eZ publish to work properly.\\
({\footnotesize http://www.php.net/manual/en/install.configure.php\#install.configure.enable-magic-quotes})
\item [--with-dom]This configures PHP to include libxml. {\footnotesize }\\
{\footnotesize (http://www.php.net/manual/en/install.configure.php\#install.configure.with-dom)}{\footnotesize \par}
\item [--with-qtdom]This configures PHP to include libqdom. It isn't up
on the PHP site with a link, but it works as --with-dom.
\item [--with-imap]This configures PHP to include imap support. This is
used by eZ mail module. This parameters require ssl support. Imap
does also have bindings to kerberos. This causes some linking problems
on RedHat Linux. The workaround for this problem is to type this command
before you compile : \texttt{\footnotesize }~\\
\texttt{\footnotesize }~\\
\texttt{\footnotesize \$ export LDFLAGS=\char`\"{}-L/usr/kerberos/lib
-lkrb5 -lgssapi\_krb5~-lpam\char`\"{}}{\footnotesize \par}
\item [--with-openssl]This will enable ssl support in PHP
\end{description}
You should also go through the web page: {\footnotesize http://www.php.net/manual/en/install.configure.php}
and make sure that there isn't other functionality you would like
to have included.


\subsubsection{Command Line}

The default is to create a command line version of PHP. Therefore
you don't need to add more configuration options for this.


\subsubsection{Apache Module}

To build an apache module you need to add:

\begin{description}
\item [--with-apxs]This compiles PHP as an apache module. {\footnotesize }\\
{\footnotesize (http://www.php.net/manual/en/install.configure.php\#install.configure.with-apxs)}{\footnotesize \par}
\end{description}

\subsubsection{Other Web Servers}

We haven't tested our software with other web servers than apache.
If you need to try out other web servers, read this document {\footnotesize http://www.php.net/manual/en/install.configure.php\#install.configure.servers}
to learn how you configure for the web server you will be using.


\subsubsection{Creating the Configuration}

Now you just have to run the {}``./configure'' program with the
apropriate configuration directives which we discussed in the preceeding
sections, for an apache module you'd do the following:\\


\texttt{\footnotesize ~~~\# ./configure -{}-enable-trans-sid -{}-with-mysql}~\\
\texttt{\footnotesize ~~~~-{}-with-apxs} \foreignlanguage{english}{}\texttt{\footnotesize -{}-with-dom
-{}-with-qtdom}~\\
{\footnotesize \par}

Remember that to compile a script/cgi version you'd need to change
that line to:\\


\texttt{\footnotesize ~~~~\# ./configure -{}-enable-trans-sid
-{}-with-mysql}~\\
\texttt{\footnotesize ~~~~-{}-with-dom -{}-with-qtdom}{\footnotesize \par}


\subsection{Compilation}

To compile you need to run the command {}``make'':\\


\texttt{\footnotesize ~~~~\# make}{\footnotesize \par}


\subsection{Installation}

To install your new PHP package you need to run the following command:\\


\texttt{\footnotesize ~~~~\# make install}{\footnotesize \par}


\subsection{Compiling the php module on RedHat 7.x, step by step}

First download the source from www.php.net. You should get a file
called something like

php-4.0.6.tar.gz\\


First, unpack the tarball:

\$ tar -xzf php-4.0.6.tar.gz\\


Now, enter the source directory

\$ cd php-4.0.6\\


Apply the kerberos workaround:

\$ export LDFLAGS=\char`\"{}-L/usr/kerberos/lib -lkrb5 -lgssapi\_krb5
-lpam\char`\"{}\\


Run the configure script:

\texttt{\footnotesize \$ ./configure -{}-with-apxs=/usr/sbin/apxs
-{}-enable-ftp -{}-with-xml -{}-with-dom -{}-enable-trans-sid -{}-with-config-file-path=/etc/httpd
-{}-with-mysql=/usr -{}-with-pgsql=/usr -{}-enable-inline-optimization
-{}-with-ttf -{}-with-qtdom -{}-with-gd -{}-enable-gd-native-ttf -{}-with-imap
-{}-includedir=/usr -{}-with-openssl=/usr -{}-with-zlib-dir=/usr -{}-with-openssl=shared,/usr}~\\
{\footnotesize \par}

Compile the module:

\texttt{\footnotesize \$ make}\\


Install the module, either automaticly or manually.\\
Manually : 

\texttt{\footnotesize \$ su}{\footnotesize \par}

\texttt{\footnotesize \# cp .libs/libphp.so /usr/lib/apache}{\footnotesize \par}

Automaticly:

\texttt{\footnotesize \$ su}{\footnotesize \par}

\texttt{\footnotesize \# make install}\\


Restart apache:

\texttt{\footnotesize \# /etc/rc.d/init.d/httpd restart}\\


Verify that everything went OK.\\
Verify that apache was able to start:

\texttt{\footnotesize \# ps ax | grep httpd}\\


Check the apache log

\texttt{\footnotesize \# tail -f 50 /var/log/httpd/error\_log}~\\
{\footnotesize \par}

IMPORTANT

When compiling php, please read chapter \ref{chptr: Troubleshooting}.
Especially, take note of chapter \ref{chptr: openssl problem on redhat}.
It might save you for hours with debugging


\chapter{Apache Configuration}

For the moment we have only one solution for configuring apache, and
that is using two virtual hosts.


\section{Dual Virtual Host\label{sec: dual virtualhost}}


\subsection{Configuring Through httpd.conf}

This set up is based on having two different virtual hosts for your
administration back-end and the main site. The main site would typically
be known as {}``www.yoursite.com'' and the administration would
be {}``admin.yoursite.com''; the names are up to you, theoretically
you could have different names, for example {}``mysite.yoursite.com''
and {}``administration.mysite.com''.

The virtual host is configured through the {}``httpd.conf'' file
which is the main configuration of Apache. Following is an example
of such a host, modify it to reflect your own installation and preferences,
but before that be sure to add the {}``NameVirtualServer'' directive
to the configuration file. The directive is {}``NameVirtualServer
ip-address'' where the ip address is the address where the server
will receive requests (http://httpd.apache.org/docs/mod/core.html\#namevirtualhost).

You should consider using the utility which we have online for creating
the configuration. The URL is http://developer.ez.no/virtualhost it
will generate a setup with the latest needed information. The presented
configuration herein might be slightly outdated, so we recommend the
online tool.


\subsubsection{User Site}

\texttt{\scriptsize ~~~~\# User site}{\scriptsize \par}

\texttt{\scriptsize ~~~~<VirtualHost~your.domain.com>}{\scriptsize \par}

\texttt{\scriptsize ~~~~~~<Directory~/your/apache/documentroot/publish\_dist>}{\scriptsize \par}

\texttt{\scriptsize ~~~~~~~~Options~FollowSymLinks~Indexes~ExecCGI}{\scriptsize \par}

\texttt{\scriptsize ~~~~~~~~AllowOverride None}{\scriptsize \par}

\texttt{\scriptsize ~~~~~~</Directory>}{\scriptsize \par}

\texttt{\scriptsize ~~~~~~RewriteEngine On}{\scriptsize \par}

\texttt{\scriptsize ~~~~~~RewriteRule \textasciicircum{}/stats/store/(.{*}).gif\$}{\scriptsize \par}

\texttt{\scriptsize ~~~~~~/your/path/to/publish/ezstats/user/storestats.php
{[}S=2{]}}{\scriptsize \par}

\texttt{\scriptsize ~~~~~~\#~The~lines~above~should~appear~on~the~same}{\scriptsize \par}

\texttt{\scriptsize ~~~~~~\#~line~in~your~configuration~file!}{\scriptsize \par}

\texttt{\scriptsize ~~~~~~RewriteRule~\textasciicircum{}/filemanager/filedownload/({[}\textasciicircum{}/{]}+)/(.{*})\$}{\scriptsize \par}

\texttt{\scriptsize ~~~~~~/your/apache/documentroot/publish\_dist/ezfilemanager/files/\$1}{\scriptsize \par}

\texttt{\scriptsize ~~~~~~{[}T=\char`\"{}application/oct-stream\char`\"{},S=1{]}}{\scriptsize \par}

\texttt{\scriptsize ~~~~~~\#~The~lines~above~should~appear~on~the~same}{\scriptsize \par}

\texttt{\scriptsize ~~~~~~\#~line~in~your~configuration~file!}{\scriptsize \par}

\texttt{\scriptsize ~~~~~~RewriteRule !\textbackslash{}.(gif|css|jpg|png)\$
/your/apache/documentroot/publish\_dist/index.php}{\scriptsize \par}

\texttt{\scriptsize ~~~~~~ServerAdmin your.e-mail@address}{\scriptsize \par}

\texttt{\scriptsize ~~~~~~DocumentRoot /your/apache/documentroot/publish\_dist}{\scriptsize \par}

\texttt{\scriptsize ~~~~~~ServerName your.domain.com}{\scriptsize \par}

\texttt{\scriptsize ~~~~</VirtualHost>}{\scriptsize \par}


\subsubsection{Admin Site}

\texttt{\footnotesize ~~\# Admin site }{\footnotesize \par}

\texttt{\footnotesize ~~<VirtualHost admin.yourdomain.org>}{\footnotesize \par}

\texttt{\footnotesize ~~~~<Directory /your/apache/documentroot/publish\_dist>}{\footnotesize \par}

\texttt{\footnotesize ~~~~~~Options FollowSymLinks Indexes ExecCGI}{\footnotesize \par}

\texttt{\footnotesize ~~~~~~AllowOverride None}{\footnotesize \par}

\texttt{\footnotesize ~~~~~~RewriteEngine On}{\footnotesize \par}

\texttt{\footnotesize ~~~~~~RewriteRule !\textbackslash{}.(gif|css|jpg|png)\$
/your/apache/documentroot/publish\_dist/index\_admin.php}{\footnotesize \par}

\texttt{\footnotesize ~~~~</Directory>}{\footnotesize \par}

\texttt{\footnotesize ~~~~ServerAdmin your\_mail@domain.no}{\footnotesize \par}

\texttt{\footnotesize ~~~~DocumentRoot /your/apache/documentroot/admin}{\footnotesize \par}

\texttt{\footnotesize ~~~~ServerName admin.yourdomain.org}{\footnotesize \par}

\texttt{\footnotesize ~~</VirtualHost>}~\\
{\footnotesize \par}

The format of the {}``httpd.conf'' file is covered at http://httpd.apache.org/docs/
for a complete understanding of the above information you'll need
to read that documentation.

The directory {}``/your/apache/documentroot/publish\_dist'' is the
directory where you extracted eZ publish.


\subsubsection{Error Checking}

You can check that everything is correct with your rewrite rules by
running {}``apache -s'', which will check for virtual hosts. There
should also be an error log (consult the apache documentation) which
you can read to check for errors.


\subsubsection{Explanation of the Rewrite Rules}

The rewrite rules do the following:\\


\texttt{\footnotesize ~~~~RewriteRule \textasciicircum{}/filemanager/filedownload/({[}\textasciicircum{}/{]}+)/(.{*})\$}{\footnotesize \par}

\texttt{\footnotesize ~~~~/your/apache/documentroot/publish\_dist/ezfilemanager/files/\$1}{\footnotesize \par}

\texttt{\footnotesize ~~~~{[}T=\char`\"{}application/oct-stream\char`\"{},S=1{]}}~\\
{\footnotesize \par}

This says that everything served from {}``/filemanager/filedownload/''
should be redirected to fetch information from {}``publish\_dist/ezfilemanager/files''.
In other words, when people downloads a file from the filemanager,
the file is served from the directory specified in the second part.

The {}``\^{ }'' just after {}``RewriteRule'' says that evertything
which starts with this, in other words it is a start of line marker.
When working with an URL that is from the root of your site, ie. the
part from the first slash after your domain name.

The {}``\$'' sign is used to mark the end of line, in order to remember
the full line.

The part \texttt{\scriptsize {}``{[}T=\char`\"{}application/oct-stream\char`\"{},S=1{]}''}
means that everything which is matched shall be of the specific mime
type ({}``application/oct-stream'', ie. binary download). The {}``S=1''
part means that if we match this rule, we should skip one rule ahead
before trying to match again.

The next rewrite rule:\\


\texttt{\footnotesize ~~~~RewriteRule !\textbackslash{}.(gif|css|jpg|png)\$
/your/apache/documentroot/publish\_dist/index.php}~\\
{\footnotesize \par}

is found in both sites (admin and user). This means that every file,
except gif, css, jpg and png (and files matched against the previous
rule when in the user site) should be redirected to the file in the
second part, ie. the index.php or index\_admin.php file. The reason
for this is that we don't want anyone trying to get direct access
to anything which might be sensitive, or revealing about the site's
operation.

If you compiled PHP with magic quotes; or other software relies on
PHP using magic quotes you can add the following line into each virtual
host section:\\


\texttt{\footnotesize ~~~~php\_flag magic\_quotes\_gpc off}~\\
{\footnotesize \par}

\emph{NOTE: It isn't possible to use the form http://mysite.com/admin
at all; since the admin module assumes that the url {}``/'' is the
start of the admin pages. If you do change eZ publish code in order
to do this anyway; please send the code to bf@ez.no for future inclusion.
The only correct way to access the admin site is through its own virtual
host address.}


\subsection{Configuring Through .htaccess}

Instead of using httpd.conf and rewrite rules in the virtual hosts,
you can also do the rewrite rules in the .htaccess filesm directory
specific configuration files.

\emph{Note: You must set up apache to accept this. You still need
two domains for this operation!}


\subsubsection{User Site}

In your main directory (/path/to/index.php/) create a file called
\char`\"{}.htaccess\char`\"{} containing the following text:\\


\texttt{\footnotesize ~~~~php\_flag magic\_quotes\_gpc off}{\footnotesize \par}

\texttt{\footnotesize ~~~~Options FollowSymLinks Indexes ExecCGI}{\footnotesize \par}

\texttt{\footnotesize ~~~~RewriteEngine On} \foreignlanguage{english}{}

\texttt{\footnotesize ~~~~RewriteRule \textasciicircum{}/stats/store/(.{*}).gif\$} \foreignlanguage{english}{}

\texttt{\footnotesize ~~~~/publish/ezstats/user/storestats.php
{[}S=2{]}} \foreignlanguage{english}{}

\texttt{\footnotesize ~~~~\# The two lines above should appear
on the} \foreignlanguage{english}{}

\texttt{\footnotesize ~~~~\# same line in your configuration file.}{\footnotesize \par}

\texttt{\footnotesize ~~~~RewriteRule \textasciicircum{}/filemanager/filedownload/({[}\textasciicircum{}/{]}+)/(.{*})\$}{\footnotesize \par}

\texttt{\footnotesize ~~~~/path/to/website/ezfilemanager/files/\$1}{\footnotesize \par}

\texttt{\footnotesize ~~~~{[}T=\char`\"{}application/oct-stream\char`\"{},S=1{]}}{\footnotesize \par}

\texttt{\footnotesize ~~~~\# The three lines above should appear
on the}{\footnotesize \par}

\texttt{\footnotesize ~~~~\# same line in your configuration file!}{\footnotesize \par}

\texttt{\footnotesize ~~~~RewriteRule !\textbackslash{}.(gif|css|jpg|png)\$
/path/to/website/index.php}{\footnotesize \par}


\subsubsection{Admin Site}

In your admin subdomain home directory, create a file with the following
text:\\


\texttt{\footnotesize ~~~~php\_flag magic\_quotes\_gpc off}{\footnotesize \par}

\texttt{\footnotesize ~~~~RewriteEngine On}{\footnotesize \par}

\texttt{\footnotesize ~~~~RewriteRule !\textbackslash{}.(gif|css|jpg|png)\$
/path/to/website/index\_admin.php}~\\
{\footnotesize \par}


\chapter{eZ publish Installation}


\section{Program Files}

The next step is to install the eZ publish package in your document
root directory. First you need to unpack the software in a temporary
directory:\\


\texttt{\footnotesize ~~~~\# cd /tmp}{\footnotesize \par}

\texttt{\footnotesize ~~~~\# tar zxvf /path/to/ezpublish-2.0.tar.gz}~\\
{\footnotesize \par}

The next step is to move the files to your document root:\\


\texttt{\footnotesize ~~~~\# mv /tmp/publish\_dist /your/apache/documentroot}~\\
{\footnotesize \par}

When all this is done you need to tell eZ publish a little about the
site you're running. You'll need to edit the {}``site.ini'' file
which you will find in the document root:\\


\texttt{\footnotesize ~~~~\# cd /your/apache/documentroot}{\footnotesize \par}

\texttt{\footnotesize ~~~~\# vi site.ini}~\\
{\footnotesize \par}

Instead of vi you can use your preferred text editor. You'll need
to add information about the username, hostname and password of your
database. More information on what you can do with {}``site.ini''
can be found in the {}``eZ publish Customisation Guide''.

The next important step is to run the script {}``secure\_modfix.sh''.
\\


\texttt{\footnotesize ~~~~\# ./secure\_modfix.sh}~\\
{\footnotesize \par}

Then you need to create cache directories:\\


\texttt{\footnotesize ~~~~\# su}~\\
\texttt{\footnotesize ~~~~\# ./secure\_cachefix.sh {[}package\_user{]}
{[}apache\_process\_group{]}}~\\
{\footnotesize \par}

Where \texttt{\footnotesize {[}package\_user{]}} is the web admin
user and \texttt{\footnotesize {[}apache\_process\_group{]}} is the
group of the running apache process.


\section{Database}

Some people might prefer to use phpMyAdmin (http://www.phpwizard.net/projects/phpMyAdmin/)
for most of this part; we can not help you with installation of that
program, though.


\subsection{First Time Installation}

Now you need to create a database in MySQL, the default name we use
is publish, but you can change that to whatever pleases you.\\


\texttt{\footnotesize ~~~~\# mysqladmin create publish}~\\
{\footnotesize \par}

Add a publish user in MySQL. To add a user you can use the MySQL client
to log on to mysql and then create the user:\\


\texttt{\footnotesize ~~~~\# mysql > grant all on publish.{*}
to publish@localhost}{\footnotesize \par}

\texttt{\footnotesize ~~~~identified by \char`\"{}secret\char`\"{};}\texttt{}~\\


where secret is your password. Then you need to add the default eZ
publish data into your newly created database:\\


\texttt{\footnotesize ~~~~\# mysql -uroot -p publish < sql/data.sql}{\footnotesize \par}


\subsubsection{Adding Pre-Defined Data}

If you want to add the pre-defined data of the distribution you shouldn't
add any data manually to the site before executing these commands.

First we need to add files and images which are needed by the database.\\


\texttt{\footnotesize ~~~~\# tar zpxvf data.tar.gz}~\\
{\footnotesize \par}

Then we need to run {}``secure\_modfix.sh'' to make sure that everything
is readable.\\


\texttt{\footnotesize ~~~~\# ./secure\_modfix.sh}~\\
{\footnotesize \par}

Then we need to send the SQL data into the database:\\


\texttt{\footnotesize ~~~~\# mysql -upublish -ppublish publish
< sql/data.sql}~\\
{\footnotesize \par}

Finally we run clearcache to make sure that everything presented is
cached correctly:\\


\texttt{\footnotesize ~~~~\# ./clearcache\_secure.sh}{\footnotesize \par}


\subsection{Updating the Installation}

This section is for users who are updating from a previous version
of eZ publish. There should be several files ending with {}``.sql''
in the directory {}``updates''. Run the files needed to update your
version to the current. You need to apply all the updates for every
version since your version.


\chapter{Now What?}

After installing eZ publish you can test your site through the URL
\texttt{\footnotesize http://www.yoursite.com/} and you can administrate
your site from the URL \texttt{\footnotesize http://admin.yoursite.com/},
of course, if you did anything different the names of the admin and
the public site might be different.

\emph{NOTE:} The default user name and password for your site will
be admin/publish. Remember to change the password.

The next manual you should read is the {}``eZ publish Customisation
Guide'', it tells you how to configure the software to use the functionality
you want, as well as how you change the templates to suit your needs.

When you're finished with the design and the initial testing you can
head over to \texttt{\footnotesize http://zez.org/} for articles about
community building as well as programming, or you can visit \texttt{\footnotesize http://developer.ez.no}
for updates, articles about eZ publish and how to work with it, as
well as keeping abreast of new developments.


\section{Post Install Checklist}

\begin{enumerate}
\item Does Apache run?
\item Does PHP run/work as an Apache module?
\item Does MySQL run?
\item Can you access your virtual hosts at all?
\item Does the user site work?
\item Does the admin site work?
\item Consider this: all eZ publish sites has an admin site, perhaps you
should call the admin host something different than admin?
\item Check that you've downloaded and read the configuration manual. A
quick tip is to read through the file {}``site.ini'' and change
any e-mail addresses, passwords etc. to fit your own choices.
\item Log in on your admin site (\texttt{\footnotesize http://admin.yoursite.com/}).
You will be presented with a page which will list any install problems.
If any problems appear read the error message presented and follow
any instructions. If that fails, read the FAQ. Then go to \texttt{\footnotesize http://developer.ez.no}
and search the forum for anyone who have had the same problem. Also
check the bug list for any open bugs covering your problem. Finally
you should register to the mailing list and try asking for help there.
\item If everything is okay go to the {}``user'' module and change the
e-mail address of the site administrator immediatly.
\item Change the password of the administration user to something only you
know.
\item Start browsing the public part of your site, just to check that everything
is working; some of the articles supplied as default will inform you
about features of the software.
\item Check that ImageMagick is working. Try to upload an image to your
site.
\end{enumerate}

\chapter{\label{chptr: Troubleshooting}Troubleshooting}


\section{Problems During Installation}


\subsection{FreeBSD 4.2 and libxml2}

The current version (2.2.11) installs itself as /usr/local/lib/libxml2.a|so
and goes unrecognized by configure (PHP). Link the files to /usr/local/lib/libxml.a|so.


\subsection{Missing Compiler/Can not Compile (C++/C)}

When compiling php and other support programs (like ImageMagick) you
need the GCC compiler. It is recommended that you use the GCC compiler
which was shipped with your Linux distro/unix system. In the introduction
(see chapter \ref{chptr: introduction}) it listed some sites where
you can download pre-compiled versions of software for some different
Unix versions. Please note that you must compile php on your own.


\subsection{I am getting linking errors when trying to build PHP}

The PHP module you have compiled will be linked agains kerberos. This
causes some linking problems on RedHat Linux. The workaround for this
problem is to type this command before you compile :

\texttt{\footnotesize \$ export LDFLAGS=\char`\"{}-L/usr/kerberos/lib
-lkrb5 -lgssapi\_krb5 -lpam\char`\"{}}{\footnotesize \par}


\section{Problems After Installation}


\subsection{Permission Denied}

\texttt{\small Warning: fopen(\char`\"{}site.ini\char`\"{},\char`\"{}r+\char`\"{})}{\small \par}

\texttt{\small Permission denied in classes/INIFile.php on line 80}{\small \par}

If you get this error message you need to run the {}``secure\_modfix.sh''
script.


\subsection{Can not see Images}

ImageMagick is not working, make sure that it is working by using
the command line command {}``convert''.


\subsection{Warning about Temp Directory}

If you get any such warning you need to set the temp directory in
php.ini.


\subsection{\label{chptr: openssl problem on redhat}After installing my new
php module, apache dies immediately.}

RedHat as released new versions of the openssl packages for RedHat
7.x. If these erratas is installed before you compile php, your php
module will be linked agains these. This will however brake mod\_ssl,
which is linked to the old openssl libraries. There is two different
ways to fix this:\\
Uninstall mod\_ssl:

\# rpm -e mod\_ssl\\
\\
Or you may download the apache source rpm from redhat. Then recompile
and install it.\\


If this doesn't help, look for clues in /var/log/httpd/error\_messages


\chapter{Installing on RAQ 3\label{chptr: raq 3}}

\emph{Installing ezPublish on raq3 without messing up the GUI or voiding
the warranty.}

This is untested by eZ systems, and we provide this {}``as is''
without any form of guarantee or endorsement, either explicitly or
implicitly.

First, add the domain into the DNS, but do not create a virtual site.

Log in by telnet (install SSH unless you are desperate to get hacked).

Put the publish files in the directory you want to use, I used /home/sites/extrasites/mysite/web

Install mysql 3.23 or later by rpm, there is one out there. MySQL
(http://www.mysql.com) version 3.23 or later if you want to compile

Now you need to create a database in MySQL, the default name we use
is publish, but you can change that to whatever pleases you.\\


\texttt{\footnotesize ~~~~\# mysql -uroot -p publish < sql/publish.sql}~\\
{\footnotesize \par}

Add a publish user in MySQL. To add a user you can use the MySQL client
to log on to mysql and then create the user:\\


\texttt{\footnotesize ~~~~\# mysql>grant all on publish.{*} to
publish@localhost}{\footnotesize \par}

\texttt{\footnotesize ~~~~identified by \char`\"{}secret\char`\"{};}~\\
{\footnotesize \par}

where secret is your password. Then you need to add the default eZ
publish data into your newly created database:\\


\texttt{\footnotesize ~~~~\# mysql -uroot -p publish < sql/publish.sql}~\\
{\footnotesize \par}

Then get:

\begin{itemize}
\item http://www.freesoftware.com/pub/infozip/zlib/ (zlib.tar.gz)
\item http://www.boutell.com/gd (gd-1.8.4.tar.gz)
\item ftp://ftp.uu.net/graphics/jpeg/jpegsrc.v6b.tar.gz (jpegsrc.v6b.tar.gz)
\item http://www.php.net (php-4.0.4pl1.tar.gz)
\end{itemize}
Delete all gd.h files on your system. You can find them using:\\


\texttt{\footnotesize ~~~~\# find / -name gd.h}~\\
{\footnotesize \par}

If there are more than one, then delete all of them.

Now add the following line to the /etc/ld.so.conf file:\\


\texttt{\footnotesize ~~~~/usr/local/lib}~\\
{\footnotesize \par}

Save the file, and run:\\


\texttt{\footnotesize ~~~~\# /sbin/ldconfig}~\\
{\footnotesize \par}

This was an important part, because Apache needs this dir to find
the correct modules.

Extract the zlib archive:\\


\texttt{\footnotesize ~~~~\# tar -zxvf zlib.tar.gz \# cd zlib-1.1.3}~\\
{\footnotesize \par}

And install it:\\


\texttt{\footnotesize ~~~~\# ./configure -{}-shared} \foreignlanguage{english}{}

\texttt{\footnotesize ~~~~\# make }{\footnotesize \par}

\texttt{\footnotesize ~~~~\# make install}~\\
{\footnotesize \par}

Now install the JPEG-6b, doing the following:\\


\texttt{\footnotesize ~~~~\# tar -zxvf jpegsrc.v6b.tar.gz}{\footnotesize \par}

\texttt{\footnotesize ~~~~\# cd jpeg-6b}{\footnotesize \par}

\texttt{\footnotesize ~~~~\# ./configure -{}-enable-shared}{\footnotesize \par}

\texttt{\footnotesize ~~~~\# make}{\footnotesize \par}

\texttt{\footnotesize ~~~~\# make install}~\\
{\footnotesize \par}

Install the PNG library\\


\texttt{\footnotesize ~~~~\# wget http://www.libpng.org/pub/png/src/libpng-1.0.9.tar.gz}~\\
{\footnotesize \par}

Then compile the package.

Get Imagemagick ImageMagick (http://www.imagemagick.org/) newest version
Download and then:\\


\texttt{\footnotesize ~~~~\# tar -zxvf Imagemagick-xxx}{\footnotesize \par}

\texttt{\footnotesize ~~~~\# cd Imagemagick-xxx}{\footnotesize \par}

\texttt{\footnotesize ~~~~\# ./configure}{\footnotesize \par}

\texttt{\footnotesize ~~~~\# make}{\footnotesize \par}

\texttt{\footnotesize ~~~~\# make install}~\\
{\footnotesize \par}

Then go one directory back, and extract the GD archive using:\\


\texttt{\footnotesize ~~~~\# tar -zxvf gd-xxx}{\footnotesize \par}

\texttt{\footnotesize ~~~~\# cd gd-xxx}~\\
{\footnotesize \par}

Now edit the Makefile (using vi or pico) and check which modules you
want. I removed the Freetype Library (-DHAVE\_LIBFREETYPE / -lfreetype).
After making the changes save the file and go back to the shell. Now
compile GD:\\


\texttt{\footnotesize ~~~~\# make} \foreignlanguage{english}{\texttt{\footnotesize }}{\footnotesize \par}

\texttt{\footnotesize ~~~~\# make install}\texttt{\small }~\\
{\small \par}

If this is giving any errors, just remove the modules you don't have
(but don't remove the JPEG lib - we need that one ! :)) )

Now go back one dir, and extract PHP4:\\


\texttt{\footnotesize ~~~~}\texttt{\small \# tar -zxvf php-4.0.4pl1.tar.gz}{\small \par}

\texttt{\footnotesize ~~~~}\texttt{\small \# cd php-4.0.4pl1}~\\
{\small \par}

First remove any cache:\\


\texttt{\footnotesize ~~~~\# rm config.cache}{\footnotesize \par}

\texttt{\footnotesize ~~~~\# make clean}{\footnotesize \par}

\texttt{\footnotesize ~~~~\#./configure -{}-with-xml -{}-with-mysql
\textbackslash{}}{\footnotesize \par}

\texttt{\footnotesize ~~~~-{}-with-apxs=/usr/sbin/apxs \textbackslash{}}{\footnotesize \par}

\texttt{\footnotesize ~~~~-{}-with-system-regex \textbackslash{}}{\footnotesize \par}

\texttt{\footnotesize ~~~~-{}-with-zlib \textbackslash{}}{\footnotesize \par}

\texttt{\footnotesize ~~~~-{}-enable-safe-mode \textbackslash{}}{\footnotesize \par}

\texttt{\footnotesize ~~~~-{}-with-gdbm \textbackslash{}}{\footnotesize \par}

\texttt{\footnotesize ~~~~-{}-enable-sysvsem \textbackslash{}}{\footnotesize \par}

\texttt{\footnotesize ~~~~-{}-with-ftp \textbackslash{}}{\footnotesize \par}

\texttt{\footnotesize ~~~~-{}-with-config-file-path=/etc/httpd/conf/
\textbackslash{}}{\footnotesize \par}

\texttt{\footnotesize ~~~~-{}-with-exec-dir=/usr/sbin/httpd \textbackslash{}}{\footnotesize \par}

\texttt{\footnotesize ~~~~-{}-with-dom \textbackslash{}}{\footnotesize \par}

\texttt{\footnotesize ~~~~-{}-enable-trans-sid}{\footnotesize \par}

\texttt{\footnotesize ~~~~\# make}{\footnotesize \par}

\texttt{\footnotesize ~~~~\# make install}~\\
{\footnotesize \par}

run /sbin/ldconfig again.

Apache: (Your milage may vary, be wary of paths)

edit /etc/httpd/conf/httpd.conf and add the Loadmodules lines like
this: \foreignlanguage{english}{}\\


\texttt{\footnotesize ~~~~\# Extra Modules}{\footnotesize \par}

\texttt{\footnotesize ~~~~LoadModule php\_module modules/mod\_php.so}{\footnotesize \par}

\texttt{\footnotesize ~~~~LoadModule php3\_module modules/libphp3.so}{\footnotesize \par}

\texttt{\footnotesize ~~~~LoadModule perl\_module /usr/lib/apache/libperl.so}{\footnotesize \par}

\texttt{\footnotesize ~~~~LoadModule php4\_module /usr/lib/apache/libphp4.so}{\footnotesize \par}

\texttt{\footnotesize ~~~~LoadModule php4\_module lib/apache/libphp4.so}~\\
{\footnotesize \par}

\# Reconstruction of the complete module list from all available modules

\# (static and shared ones) to achieve correct module execution order.

\# {[}WHENEVER YOU CHANGE THE LOADMODULE SECTION ABOVE UPDATE THIS,
TOO{]}\\


\texttt{\footnotesize ~~~~ClearModuleList}{\footnotesize \par}

\texttt{\footnotesize ~~~~\# Extra Modules}{\footnotesize \par}

\texttt{\footnotesize ~~~~AddModule mod\_php.c}{\footnotesize \par}

\texttt{\footnotesize ~~~~AddModule mod\_php3.c}{\footnotesize \par}

\texttt{\footnotesize ~~~~AddModule mod\_perl.c}{\footnotesize \par}

\texttt{\footnotesize ~~~~AddModule mod\_php4.c}\texttt{\small }~\\
{\small \par}

Add the second line below line below the rewrite stuff, above the
<Virtualhost> directives.

NameVirtualHost 216.97.67.4 Include /etc/httpd/conf/extrasites.conf
<VirtualHost 216.97.67.4>

create this include file and in it put the apache vitual server directives
for your site.

For example:\\


\texttt{\footnotesize ~~~~\# User site}{\footnotesize \par}

\texttt{\footnotesize ~~~~<VirtualHost yourIP>}{\footnotesize \par}

\texttt{\footnotesize ~~~~~~ServerName yourdomain.org}{\footnotesize \par}

\texttt{\footnotesize ~~~~~~ServerAlias www.yourdomain.org}{\footnotesize \par}

\texttt{\footnotesize ~~~~~~<Directory /your/site/root/>}{\footnotesize \par}

\texttt{\footnotesize ~~~~~~~~Options~FollowSymLinks~Indexes~ExecCGI}{\footnotesize \par}

\texttt{\footnotesize ~~~~~~~~AllowOverride~None}{\footnotesize \par}

\texttt{\footnotesize ~~~~~~</Directory>}{\footnotesize \par}

\texttt{\footnotesize ~~~~~~RewriteEngine On}{\footnotesize \par}

\texttt{\footnotesize ~~~~~~RewriteRule !\textbackslash{}.(gif|css|jpg|png)\$
/your/site/root/index.php}{\footnotesize \par}

\texttt{\footnotesize ~~~~~~ServerAdmin your\_mail@domain.no}{\footnotesize \par}

\texttt{\footnotesize ~~~~~~DocumentRoot /your/site/root/}{\footnotesize \par}

\texttt{\footnotesize ~~~~</VirtualHost>}~\\
{\footnotesize \par}

\texttt{\footnotesize ~~~~\# Admin site}{\footnotesize \par}

\texttt{\footnotesize ~~~~<VirtualHost admin.yourdomain.org>}{\footnotesize \par}

\texttt{\footnotesize ~~~~~~<Directory /your/site/root/admin>}{\footnotesize \par}

\texttt{\footnotesize ~~~~~~~~Options~FollowSymLinks~Indexes~ExecCGI}{\footnotesize \par}

\texttt{\footnotesize ~~~~~~~~AllowOverride None}{\footnotesize \par}

\texttt{\footnotesize ~~~~~~</Directory>}{\footnotesize \par}

\texttt{\footnotesize ~~~~~~RewriteEngine On}{\footnotesize \par}

\texttt{\footnotesize ~~~~~~RewriteRule !\textbackslash{}.(gif|css|jpg|png)\$
/your/site/root/admin/index.php}{\footnotesize \par}

\texttt{\footnotesize ~~~~~~ServerAdmin your\_mail@domain.no}{\footnotesize \par}

\texttt{\footnotesize ~~~~~~DocumentRoot /your/site/root/admin}{\footnotesize \par}

\texttt{\footnotesize ~~~~~~ServerName admin.yourdomain.org}{\footnotesize \par}

\texttt{\footnotesize ~~~~~~ServerAlias admin.yourdomain.org}{\footnotesize \par}

\texttt{\footnotesize ~~~~</VirtualHost>}~\\
{\footnotesize \par}

restart apache:\\


\texttt{\footnotesize ~~~~\# /etc/rc.d/init.d/httpd stop}~\\
{\footnotesize \par}

wait a few seconds then\\


\texttt{\footnotesize ~~~~\# /etc/rc.d/init.d/httpd start}~\\
{\footnotesize \par}

Then chown httpd.httpd {*} on both the domain and admin.domain directories
to get it to work.

If all is well, your site should work.


\section{Getting SSL to Work}

This is a bit tougher! Enable SSL for the site in your GUI. Generate
your certificates. Disable SSL in the GUI. Add this to the end of
your extrasites.conf

\texttt{\footnotesize ~~~~\#attempt to modify} \foreignlanguage{english}{\texttt{\footnotesize }}{\footnotesize \par}

\texttt{\footnotesize ~~~~SSL Listen xxx.xxx.xxx.xxx:443}{\footnotesize \par}

\texttt{\footnotesize ~~~~<VirtualHost xxx.xxx.xxx.xxx:443>}{\footnotesize \par}

\texttt{\footnotesize ~~~~~~~~ServerAdmin ubong}{\footnotesize \par}

\texttt{\footnotesize ~~~~~~~~DocumentRoot /home/sites/yoursite/web}{\footnotesize \par}

\texttt{\footnotesize ~~~~~~~~<Directory /home/sites/yoursite/web>}{\footnotesize \par}

\texttt{\footnotesize ~~~~~~~~~~~~Options FollowSymLinks
Indexes ExecCGI}{\footnotesize \par}

\texttt{\footnotesize ~~~~~~~~~~~~AllowOverride None}{\footnotesize \par}

\texttt{\footnotesize ~~~~~~~~</Directory>}{\footnotesize \par}

\texttt{\footnotesize ~~~~~~~~SSLEngine on}{\footnotesize \par}

\texttt{\footnotesize ~~~~~~~~SSLCertificateFile /home/sites/yoursite/certs/certificate}{\footnotesize \par}

\texttt{\footnotesize ~~~~~~~~SSLCertificateKeyFile /home/sites/yoursite/certs/key}{\footnotesize \par}

\texttt{\footnotesize ~~~~~~~~AddHandler server-parsed .shtml}{\footnotesize \par}

\texttt{\footnotesize ~~~~~~~~AddType text/html .shtml}{\footnotesize \par}

\texttt{\footnotesize ~~~~~~~~AddHandler cgi-wrapper .cgi}{\footnotesize \par}

\texttt{\footnotesize ~~~~~~~~AddHandler cgi-wrapper .pl}{\footnotesize \par}

\texttt{\footnotesize ~~~~~~~~RewriteEngine On}{\footnotesize \par}

\texttt{\footnotesize ~~~~~~~~RewriteRule !\textbackslash{}.(gif|css|jpg|png)\$
/home/sites/public.edge.ai/web/index.php}{\footnotesize \par}

\texttt{\footnotesize ~~~~~~~~ErrorLog /home/sites/yoursite/logs/error\_log}{\footnotesize \par}

\texttt{\footnotesize ~~~~~~~~TransferLog /home/sites/yoursite/logs/access\_log}{\footnotesize \par}

\texttt{\footnotesize ~~~~</VirtualHost>}{\footnotesize \par}

This should work. IF you can't get it, give me an email and I'll help
if I have time: chris@net.ai


\chapter{Installing on Windows }

Please download and read the Windows documentation from the eZ publish
site (http://developer.ez.no). Check the {}``File Archive'' link
under {}``Downloads'' . It covers all the steps needed for a working
installation under Windows using Apache.


\chapter{Mandrake\label{chptr: mandrake}}

Thanks to Mark Polsen for giving us this information.

This is untested by eZ systems, and we provide this {}``as is''
without any form of guarantee or endorsement, either explicitly or
implicitly.


\section{Download the Following}

Download what you need from \texttt{\footnotesize http://developer.ez.no/filemanager/list/23/}
and install the rpms!

That's all; read the manual from the start after that, and install
eZ publish as described.
\end{document}
